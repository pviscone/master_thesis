%!TEX root = ../thesis.tex

%%%%%%%%%%%%%%%%%%%%%%%%%
% Object reconstruction %
%%%%%%%%%%%%%%%%%%%%%%%%%
\message{^^J ^^J OBJECTS ^^J ^^J} % print to log
\newchap{Object \& Event Reconstruction} \label{sec:objects}
% https://twiki.cern.ch/twiki/bin/viewauth/CMS/Internal/PubDetector detector lines

This chapter explains object and event reconstruction in the CMS experiment.


%%%%%%%%%%%%%%%%%%%%%%
%   PID PRINCIPLES   %
%%%%%%%%%%%%%%%%%%%%%%
\section{Principles of particle identification} \label{sec:PID}

% FIGURE: Particle lifetimes
%!TEX root = ../thesis.tex


% FIGURE: Mass vs. lifetime
% https://www.sciencedirect.com/science/article/pii/S0146641019300109
\begin{figure*}[p!]
  \centering %\vspace{-6mm}
  \includegraphics[width=10.5cm]{fig/objects/SM_particles_masses.pdf}
  \vspace{-2mm}
  \caption{
Plot of the mass versus lifetime $\tau$ of many composite and fundamental SM particles.
The decay length is given by $L=\gamma\beta c\tau$, with $\gamma\beta=p/mc$.
Adapted from~\cite{lifetime_plot}.
  } \label{fig:lifetime}
\end{figure*}

% FIGURE: Particle identification
%!TEX root = ../thesis.tex

% FIGURE: PID principles
\begin{figure*}[p!]
  %\vspace{-3mm}
  \centering
  \includegraphics[width=0.92\textwidth]{fig/detector/CMS_detector_PID_edit.pdf}
  \caption{
Particles in the CMS detector.
Adapted from from~\cite{CMS_PID}. % David Barney.
  } \label{fig:PID}
  %\vspace{-3mm}
\end{figure*}


Figure~\ref{fig:lifetime} compares the lifetime, dividing them in different regions of stability in the context of the CMS experiment.

Each of these particles has a unique set of properties that leave a characteristic signal in one or more of the subdetector, as illustrated in \Fig{fig:PID}.


%%%%%%%%%%%%%%%%%%%%%
%   PARTICLE-FLOW   %
%%%%%%%%%%%%%%%%%%%%%
\section{Particle-flow algorithm} \label{sec:PF}
The particle-flow (PF) algorithm~\cite{PF1,PF2017} fully reconstructs the event of a pp collision with an optimized combination of all measurements of the CMS subsystems.

Tracks in the inner tracker or muon system are iteratively built from hits, using the \emph{Kalman-filter} (KF) \emph{technique}~\cite{CMS_track_reco_2006,Kalman_filtering}.


%%%%%%%%%%%%%%%%%%%%%%
%   PRIMARY VERTEX   %
%%%%%%%%%%%%%%%%%%%%%%
\section{Primary vertex} \label{sec:PV}
The algorithm locates primary vertices (PVs) of $\Pp\Pp$ collisions.
The tracks are clustered using the \emph{deterministic annealing algorithm}~\cite{deterministic_annealing}.
Finally, candidate vertices are fitted using an \emph{adaptive vertex fitter}~\cite{vertex_fitting}.
The PV with the largest sum of momenta is assumed to be the vertex of the hard scattering process~\cite{CMS_vertex,PF2017,CMS_vertex_phase2}.


%%%%%%%%%%%%%%%%%
%   ELECTRONS   %
%%%%%%%%%%%%%%%%%
\section{Electrons} \label{sec:electron}

Electrons are reconstructed by associating a track in the inner tracker to clusters of energy deposits in the ECAL.
% tracks
Candidate tracks are refitted with the \emph{Gaussian sum filter} (GSF)~\cite{GSF}.
% shower
The ECAL clusters are recombined into a so-called \emph{supercluster}.
The resolution is documented in Ref.~\cite{CMS_electron_2021,CMS_electron_calibration_2016}.

% ID
The reconstruction and identification of electrons, as well as high-energy photons, are discussed in more detail in Refs.~\cite{CMS_electron} and~\cite{CMS_electron_2021}.

The \emph{relative isolation} for electrons is computed with the so-called \emph{rho-effective-area method}~\cite{PU_substraction}:
\begin{equation} \label{eq:iso_ele}
  \Irel^{\Pe} \defeq
    \frac{\sum_\text{charged} \PT + \max\left( 0, \sum_\text{neutral} \ET
                                  - \rho A_\text{eff} \right )}{\pt^{\Pe}}. 
\end{equation}


%%%%%%%%%%%%%
%   MUONS   %
%%%%%%%%%%%%%
\section{Muons} \label{sec:muon}
Muon candidates in CMS are reconstructed as \emph{standalone muons}, \emph{tracker muons}, and/or \emph{global muons}~\cite{CMS_muon_2018}.

% RELATIVE ISOLATION
The relative isolation of a muon is defined with the so-called \emph{$\Delta\beta$-corrections}:
\begin{equation}\label{eq:iso_mu}
  \Irel^{\PGm} \defeq
    \frac{\sum_\text{charged} \pt + \max\left( 0, \sum_\text{neutral} \ET
                                  - \Delta\beta \sum_\text{charged, PU} \pt \right )}{\pt^{\PGm}}.
\end{equation}


%%%%%%%%%%%%
%   TAUS   %
%%%%%%%%%%%%
\section{Hadronically Decayed \texorpdfstring{$\PGt$}{tau} Leptons} \label{sec:tauh}
Hadronic decays of $\PGt$ leptons ($\tauh$) are reconstructed with the \emph{hadron-plus-strips} (HPS) \emph{algorithm} ~\cite{HPS1,HPS2,HPS3,DeepTau}.
About $65\%$ of hadronic $\PGt$ decays involve neutral pions, which decay promptly ($\tau=8.4\times10^{-17}\unit{s}$) to two photons $98.8\%$ of the time~\cite[p.~33]{PDG_2022}.

The $\tauh$ identification algorithm \DeepTau~\cite{DeepTau} was used to discriminate against jets that are initiated by quarks and gluons, as well as electrons and muons, see \Fig{fig:tau_jets}.

% FIGURES: tau decay
%!TEX root = ../thesis.tex

% FIGURE: tau decay pie charts
\begin{figure*}[p!]
  %\vspace{-6mm}
  \centering
  \raisebox{-0.40\height}{\includegraphics[width=0.41\textwidth,page=2]{fig/objects/SM_decay_piechart.pdf}}
  \hspace{5mm}
  \raisebox{-0.50\height}{\includegraphics[width=0.21\textwidth,page=4]{fig/objects/SM_decay_piechart.pdf}}
  \vspace{-1mm}
  \caption{
Pie charts of branching fractions.
\Left: $\PGt$ lepton decay.
\Right: Decay channels of a pair of $\PGt$ leptons.
Numbers from PDG~\cite[p.~28]{PDG_2022}.
  } \label{fig:tau_decay_piechart}
  %\vspace{-3mm}
\end{figure*}


% FIGURES: tau decay
%!TEX root = ../thesis.tex

% FIGURE: tau decay signatures
\begin{figure*}[p]
  %\vspace{-3mm}
  \centerline{
    \hspace{3mm}
    \includegraphics[width=1.05\textwidth,page=2]{fig/tau/tau_decay.pdf}
  }
  \caption{
An illustration $\tauh$ signatures.
Inspired by Ref.~\cite[p.~37]{Yuta}.
  } \label{fig:tauh_signature}
  %\vspace{-3mm}
\end{figure*}


% FIGURES: tau backgrounds
%!TEX root = ../thesis.tex

% FIGURE: tau backgrounds
\begin{figure*}[p!]
  %\vspace{-3mm}
  \def\mybr{\\[-0.5pt]\phantom{.}\hspace{5.3mm}}
  \centerline{
    \subfloat[Genuine $\PGp^-$.]{ %$\PGt^\pm \to h^\pm \PGnGt$
      \includegraphics[width=0.145\linewidth,page=1]{fig/tau/jet_tau.pdf}} \hspace{1mm}
    \subfloat[Genuine\mybr$\PGp^- \PGp^0\PGp^0$.]{ %$\PGt^\pm \to h^\pm \PGp^0\PGp^0 \PGnGt$
      \includegraphics[width=0.145\linewidth,page=2]{fig/tau/jet_tau.pdf}} \hspace{1mm}
    \subfloat[Genuine\mybr$\PGp^+ \PGp^- \PGp^+ \PGp^0$.]{ %$\PGt^\pm \to h^\pm h^\pm h^\pm \PGp^0 \PGnGt$
      \includegraphics[width=0.145\linewidth,page=4]{fig/tau/jet_tau.pdf}} \hspace{1mm}
  %}\centerline{
    \subfloat[Isolated\mybr electron.]{
      \includegraphics[width=0.145\linewidth,page=5]{fig/tau/jet_tau.pdf}} \hspace{1mm}
    \subfloat[Isolated\mybr muon.]{
      \includegraphics[width=0.145\linewidth,page=6]{fig/tau/jet_tau.pdf}} %\hspace{0.4mm}
    \subfloat[Quark or\mybr gluon jet.]{
      \includegraphics[width=0.155\linewidth,page=7]{fig/tau/jet_tau.pdf}}
  }
  \caption{
Illustration of several hadronic decay modes of the $\PGt$ leptons (a-c) and their backgrounds (d-f).
  } \label{fig:tau_jets}
  %\vspace{-3mm}
\end{figure*}



%%%%%%%%%%%%
%   JETS   %
%%%%%%%%%%%%
\section{Jets}\label{sec:jets}

% https://arxiv.org/abs/1706.04965
Jets are reconstructed with a clustering algorithm.
The jet properties must be \emph{infrared safe} and \emph{collinear safe}.
% anti-kT
The \emph{anti-$\kt$} (AK) \emph{algorithm} utilizes the metric
\begin{equation} \label{eq:antikT}
  d_{ij} = \min\left(p_\text{T,i}^{-2},p_\text{T,j}^{-2}\right)\frac{\Delta R^2_{ij}}{\Delta R_y^2},
\end{equation}
where $\Delta R_y=\sqrt{\Delta y^2+\Delta\phi^2}$ is the distance in $(y,\phi)$-space with the \emph{rapidity}
\begin{equation} \label{eq:rapidity}
  y \defeq \frac{1}{2}\ln\left(\frac{E+p_z}{E-p_z}\right).
\end{equation}
If a cluster $i$ of combined objects satisfies
\begin{equation} \label{eq:antikT_cut}
  d_{ij} > d_{i\text{B}} = p_\text{T,i}^{-2},
\end{equation}
More info is given in Ref.~\cite{PF2017,antikT}.
The AK algorithm is implemented in the \FASTJET library~\cite{fastjet1,fastjet2}, and \emph{charged-hadron subtraction} (CHS) is used.

% JEC
The CMS Collaboration determines \emph{jet energy corrections} (JECs) in several steps, which are explained in detail in Ref.~\cite{CMS-JME-10-011}.
Additionally, the jet energy and \emph{jet energy resolution} (JER) of jets in simulation are corrected to obtain better agreement between simulation and data.

% JET ID
Genuine jets are distinguished from pileup jets with the identification algorithm~\cite{jetPUID}.
Identification~\cite{jetID_2016} removes spurious jet-like features that originate from isolated noise patterns in certain HCAL regions.


%%%%%%%%%%%%%
%   B TAG   %
%%%%%%%%%%%%%
\section{Bottom quark identification} \label{sec:btagging}

% FIGURE: B TAGGING ILLUSTRATION
%!TEX root = ../thesis.tex

% FIGURE: PID
\begin{figure*}[t!]
  %\vspace{-6mm}
  \centering
  \includegraphics[width=0.33\textwidth]{fig/objects/jet_btag.pdf}
  %\vspace{-2mm}
  \caption{
B-tagging.
  } \label{fig:btagging}
  %\vspace{-8mm}
\end{figure*}


The \DeepCSV algorithm~\cite{btag_deepCSV,btag_2018} for b tagging jets. \DeepCSV is a deep neural network (DNN) with four hidden layers that is an extension of the combined secondary vertex (CSV) algorithm~\cite{btag_deepCSV,btag_2018}.


%%%%%%%%%%%
%   MET   %
%%%%%%%%%%%
% https://twiki.cern.ch/twiki/bin/view/CMSPublic/WorkBookMetAnalysis#7_7_6_MET_Corrections
\section{Missing transverse energy}\label{sec:met}

% FIGURE: MET
%!TEX root = ../thesis.tex

% FIGURE: MET
\begin{figure*}[t]
  %\vspace{-6mm}
  \centering
  \includegraphics[width=0.39\textwidth]{fig/objects/transverse_plane.pdf}
  %\vspace{-2mm}
  \caption{
Illustration of missing transverse energy (MET) with a muon track (red), and energy deposits in the calorimeters.
  } \label{fig:MET}
  %\vspace{-2mm}
\end{figure*}


The vectorially sum of all particles gives the missing momentum
\begin{equation} \label{eq:MET}
  \ptvecmiss = - \sum_{\text{i}}\myvec{p}_\text{T}^{\,i}.
\end{equation}
This vector, or its length, is often referred to as the \emph{missing transverse momentum}, or \emph{missing transverse energy} (MET).
Several corrections are applied~\cite{PF2017}.
More details can be found in Refs.~\cite{CMS-PAS-JME-16-004} and~\cite{CMS-PAS-JME-17-001}. 

