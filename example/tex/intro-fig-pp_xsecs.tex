%!TEX root = ../thesis.tex

% FIGURE: pp: hard process
\begin{figure*}[t]
  \vspace{-8mm}
  \centering
  \includegraphics[height=0.18\textheight]{fig/feynman/proton-proton_collision_generic_hard_process.pdf} \qquad
  \includegraphics[height=0.18\textheight]{fig/feynman/proton-proton_Drell-Yan.pdf} %\vspace{-2mm}
  \caption{
\Left:~Feynman diagrams of a pp collision with a generic hard process represented by a blob in the middle. The  groups of three lines coming from the left represent the proton with its substructure. The hard process is effectively a collision of two partons $i=1,2$ emerging from each proton and carrying a fraction $x_i$ of their mother proton's total momentum $p_i$~\cite{parton_model_pp}. The rest of the proton becomes debris, and continues along the beamline. 
\Right:~Simple example of a hard process: Drell-Yan at tree level, which involves a virtual photon $\gamma^*$ or $\PZ$ boson from quark-antiquark annihilation creating a lepton-antilepton pair.
  } \label{fig:hard_process}
  %\vspace{-3mm}
\end{figure*}


% FIGURE: pp cross sections
\begin{figure*}[p!]
  \vspace{-14mm}
  \centering
  \includegraphics[width=0.55\textwidth,clip,trim=0mm 0mm 0mm 12mm]{fig/intro/LHC_proton_cross_sections_MSTW2008.pdf}
  \vspace{-2mm}
  \caption{
Plot of the most common processes in proton-proton collisions and their cross sections numerically calculated at NLO or NNLO accuracy in perturbation theory of QCD with MSTW2008 PDFs~\cite{LHC_pdfs}. %(NLO or NNLO)
Taken from~\cite{pp_cross_sections}.
  } \label{fig:pp_cross_sections}
  %\vspace{-1mm}
\end{figure*}


% FIGURE: pp cross section CMS
\begin{figure*}[p!]
  %\vspace{-8mm}
  \centering
  \centerline{
    \includegraphics[width=1.08\textwidth]{fig/intro/SigmaNew_v8.pdf}
  }
  \vspace{1mm}
  \caption{
Predicted and measured cross sections for a variety of SM processes wherein one or more top quarks are produced in pp collisions at center-of-mass energies of $\sqrt{s}=7$, 8, and 13\TeV, as measured by the CMS Collaboration.
The $x$ axis denotes the process. The cross section for the production of a pair of top quarks ($\ttbar$) is shown inclusively, as well as separately for production with additional jets (j), bottom quark ($\PQb$), or charm quark ($\PQc$). ``s-ch'' and ``t-ch'' refer to $s$-channel or $t$-channel production of a single top quark, respectively.
Taken from~\cite{pp_cross_sections_CMS}.
  } \label{fig:pp_cross_sections_CMS}
  \vspace{-7mm}
\end{figure*}

