%!TEX root = ../thesis.tex
% Author: Izaak Neutelings (February 2023)
% Description: Mimicking common CMS TDR macros & particle pennames
% Sources:
%   https://gitlab.cern.ch/tdr/utils/-/blob/master/general/hepparticles.sty
%   https://gitlab.cern.ch/tdr/utils/-/blob/master/general/heppennames2.sty
%   https://gitlab.cern.ch/tdr/utils/-/blob/master/general/ptdr-definitions.sty

% CMS TDR: text
\newcommand{\etal}   {\mbox{et al.}\xspace} %et al. - no preceding comma
\newcommand{\ie}     {\mbox{i.e.}\xspace}
\newcommand{\eg}     {\mbox{e.g.}\xspace}
\newcommand{\etc}    {\mbox{etc.}\xspace}

% CMS TDR: common macros
\newcommand{\PNfix}   {\hspace{-.04em}}
\newcommand*{\DOI}[1]{\href{http://dx.doi.org/#1}{\texttt{doi:#1}}} % CHECK REFERENCE 10.1103 !!!
\providecommand{\NA}{\ensuremath{\text{---}}}
\providecommand{\cmsTable}[1]{\resizebox{\textwidth}{!}{#1}}

% CMS TDR: units
\newcommand{\unit}[1]{{\ensuremath{\text{\,#1}}}\xspace}
\newcommand{\mus}    {{\ensuremath{\,\text{\textmu s}}}\xspace} %\upmu
\newcommand{\mum}    {{\ensuremath{\,\text{\textmu m}}}\xspace}
\newcommand{\cm}     {{\ensuremath{\,\text{cm}}}\xspace}
\newcommand{\MeV}    {{\ensuremath{\,\text{Me\hspace{-.08em}V}}}\xspace}
\newcommand{\GeV}    {{\ensuremath{\,\text{Ge\hspace{-.08em}V}}}\xspace}
\newcommand{\TeV}    {{\ensuremath{\,\text{Te\hspace{-.08em}V}}}\xspace}
\newcommand{\fbinv}  {{\mbox{\ensuremath{\,\text{fb}^\text{$-$1}}}}\xspace}

% CMS TDR: SOFTWARE PROGRAMS
\newcommand{\GEANTfour} {{\textsc{Geant4}}\xspace}
\newcommand{\GEANTthree} {{\textsc{geant3}}\xspace}
\newcommand{\GEANT} {{\textsc{geant}}\xspace}
\newcommand{\FASTJET} {{\textsc{FastJet}}\xspace}
\newcommand{\FEWZ} {{\textsc{fewz}}\xspace}
\newcommand{\Toppp} {\textsc{Top$++$}\xspace}
\newcommand{\HERWIG} {{\textsc{herwig}}\xspace}
\newcommand{\PYTHIA} {{\textsc{pythia}}\xspace}
\newcommand{\MADGRAPH} {\textsc{MadGraph}\xspace}
\newcommand{\aMCATNLO}{a\textsc{mc@nlo}\xspace}
\newcommand{\MCATNLO} {\textsc{mc@nlo}\xspace}
\newcommand{\MGvATNLO}{\MADGRAPH{}5\_a\MCATNLO}
\newcommand{\POWHEG} {{\textsc{powheg}}\xspace}
%\newcommand{\TAUOLA} {\textsc{tauola}\xspace}
\newcommand{\DeepTau} {{\textsc{DeepTau}}\xspace}
\newcommand{\DeepCSV} {{\textsc{DeepCSV}}\xspace}
%\newcommand{\Djet}{\ensuremath{D_\text{jet}}\xspace} % DeepJet
%\newcommand{\DeepTauVSe}{\texttt{DeepTau2017v2p1VSe}\xspace}
%\newcommand{\DeepTauVSmu}{\texttt{DeepTau2017v2p1VSmu}\xspace}
%\newcommand{\DeepTauVSjet}{\texttt{DeepTau2017v2p1VSjet}\xspace}

% CMS TDR PARTICLE PENNAMES
% https://gitlab.cern.ch/tdr/utils/-/blob/master/general/heppennames2.sty
% grep '\\PZ' */*.tex *.tex
% sed -e 's/\\PZ/\\PZ/g' -i '' *.tex */*.tex # macOS
% https://mirror.foobar.to/CTAN/macros/latex/contrib/was/upgreek.pdf
% https://ctan.org/pkg/fntguide

% UNSLANT small greek letters to make them look straight for particle names
% https://tex.stackexchange.com/questions/145926/upright-greek-font-fitting-to-computer-modern
% https://tex.stackexchange.com/questions/236915/adjust-custom-made-upright-greek-letters-when-used-in-subscripts
\usepackage{scalerel}
\newsavebox{\foobox}
\newcommand{\slantbox}[2][0]{\mbox{%
        \sbox{\foobox}{#2}%
        \hskip\wd\foobox
        \pdfsave
        \pdfsetmatrix{1 0 #1 1}%
        \llap{\usebox{\foobox}}%
        \pdfrestore
}}
\newcommand\unslant[2][-.25]{%
  \mkern1.2mu%
  \ThisStyle{\slantbox[#1]{$\SavedStyle#2$}}%
  \mkern-1.2mu%
}
\newcommand\unslantlong[2][-.25]{% % for long letter like mus
  \mkern-0.2mu%
  \ThisStyle{\slantbox[#1]{$\SavedStyle#2$}}%
  \mkern-0.8mu%
}
%\newcommand{\upalpha}{\unslant\alpha}
%\newcommand{\upgamma}{\unslant\gamma}
%\newcommand{\upeta}{\unslant\eta}
\newcommand{\upmu}{\unslantlong\mu}
\newcommand{\upnu}{\unslant\nu}
\newcommand{\uppi}{\unslant\pi}
\newcommand{\uprho}{\unslant\rho}
\newcommand{\uptau}{\unslant\tau}
%\newcommand{\upphi}{\unslantlong\phi}
%\newcommand{\upchi}{\unslantlong\chi}
\newcommand{\uppsi}{\unslant\psi}
%\newcommand{\upomega}{\unslant\omega}

% CMS TDR: PENNAMES
% https://gitlab.cern.ch/tdr/utils/-/blob/master/general/heppennames2.sty
\newcommand{\Pe}{{\ensuremath{\text{e}}}\xspace} % electron
\newcommand{\PGm}{{\ensuremath{\upmu}}\xspace} % muon
\newcommand{\PGt}{{\ensuremath{\uptau}}\xspace}
\newcommand{\PGn}{{\ensuremath{\upnu}}\xspace}
\newcommand{\PGnl}{{\ensuremath{\upnu_{\ell}}}\xspace}
\newcommand{\PGne}{{\ensuremath{\upnu_{\Pe}}}\xspace}
\newcommand{\PGnGm}{{\ensuremath{\upnu_{\PGm\mkern-0.9mu}}}\xspace}
\newcommand{\PGnGt}{{\ensuremath{\upnu_{\PGt\mkern-0.9mu}}}\xspace}
\newcommand{\PAGnl}{{\ensuremath{\overline\upnu_{\ell}}}\xspace} % anti-neutrino
\newcommand{\PAGne}{{\ensuremath{\overline\upnu_{\Pe}}}\xspace}
\newcommand{\PAGnGm}{{\ensuremath{\overline\upnu_{\PGm\mkern-.9mu}}}\xspace}
\newcommand{\PAGnGt}{{\ensuremath{\overline\upnu_{\PGt\mkern-.9mu}}}\xspace}
\newcommand{\PQu}{{\ensuremath{\text{u}}}\xspace} % for u quark
\newcommand{\PQd}{{\ensuremath{\text{d}}}\xspace} % for d quark
\newcommand{\PQc}{{\ensuremath{\text{c}}}\xspace} % for c quark
\newcommand{\PQs}{{\ensuremath{\text{s}}}\xspace} % for s quark
\newcommand{\PQt}{{\ensuremath{\text{t}}}\xspace} % for t quark
\newcommand{\PQb}{{\ensuremath{\text{b}}}\xspace} % for b quark
\newcommand{\PAQu}{{\ensuremath{\bar{\text{u}}}}\xspace} % for u antiquark
\newcommand{\PAQd}{{\ensuremath{\bar{\text{d}}}}\xspace} % for d antiquark
\newcommand{\PAQc}{{\ensuremath{\bar{\text{c}}}}\xspace} % for c antiquark
\newcommand{\PAQs}{{\ensuremath{\bar{\text{s}}}}\xspace} % for s antiquark
\newcommand{\PAQt}{{\ensuremath{\bar{\text{t}}}}\xspace} % for t antiquark
\newcommand{\PAQb}{{\ensuremath{\bar{\text{b}}}}\xspace} % for b antiquark
\newcommand{\PQq}{{\ensuremath{q}}\xspace} % generic quark
\newcommand{\PGg}{{\ensuremath{\gamma}}\xspace} % photon (gamma)
\newcommand{\Pg}{{\ensuremath{g}}\xspace} % generic gluon
\newcommand{\PW}{{\ensuremath{\text{W}}}\xspace} % W boson
\newcommand{\PZ}{{\ensuremath{\text{Z}}}\xspace} % Z boson
\newcommand{\PH}{{\ensuremath{\text{H}}}\xspace} % Higgs boson
\newcommand{\Zg}{{\ensuremath{\PZ\kern-0.5pt/\kern-0.5pt\gamma^*}}\xspace} % photon / Z
\newcommand{\PGp}{\ensuremath{\uppi}\xspace} % pion
\newcommand{\PGr}{\ensuremath{\uprho}\xspace} % rho
\newcommand{\Pa}{\ensuremath{\text{a}}\xspace} % a_1
\newcommand{\PK}{\ensuremath{\mathrm{K}}\xspace} % kaon
\newcommand{\PAK}{\ensuremath{\overline{\mathrm{K}}}\xspace} % kaon
\newcommand{\Pp}{\ensuremath{\mathrm{p}}\xspace} % proton
\newcommand{\Pn}{\ensuremath{\mathrm{n}}\xspace} % neutron
\newcommand{\PD}{\ensuremath{\mathrm{D}}\xspace} % D meson
\newcommand{\PB}{\ensuremath{\mathrm{B}}\xspace} % B meson
\newcommand{\PAB}{\ensuremath{\overline{\mathrm{B}}}\xspace} % B meson
\newcommand{\PJGy}{\ensuremath{\mathrm{J}\mspace{-2mu}/\mspace{-2mu}\uppsi}\xspace} %  J/psi meson
\newcommand{\PGU}{\ensuremath{\Upsilon}\xspace} % Upsilon

% CMS TDR: common particle combinations
\newcommand{\bsln} {\ensuremath{\PQb\to\PQc\ell\PAGnl}\xspace} % b -> c ell nu
\newcommand{\bstn} {\ensuremath{\PQb\to\PQc\PGt\PGnGt}\xspace} % b -> s tau nu
\newcommand{\ttbar}{{\ensuremath{\PQt\PAQt}}\xspace}
\newcommand{\bbbar}{{\ensuremath{\PQb\PAQb}}\xspace}
\newcommand{\EE}   {{\ensuremath{\Pe^+\Pe^-}}\xspace}
\newcommand{\MM}   {{\ensuremath{\PGm^+\PGm^-}}\xspace}
\newcommand{\TT}   {{\ensuremath{\PGt^+\PGt^-}}\xspace}
\newcommand{\LL}   {{\ensuremath{\ell^+\ell^-}}\xspace}

% CMS TDR: common particle physics symbols
\newcommand{\lumi}   {{\ensuremath{\mathcal{L}}}\xspace}
\newcommand{\DR}     {{\ensuremath{\Delta R}}\xspace}
\newcommand{\ET}     {\ensuremath{E_\text{T}}\xspace}
\newcommand{\kt}     {\ensuremath{k_\text{T}}\xspace}
\newcommand{\pt}     {\ensuremath{p_\text{T}}\xspace}
\newcommand{\ptvec}  {{\ensuremath{\myvec{p}_\text{T}}}\xspace}
\newcommand{\ptmiss} {\ensuremath{\pt^\text{miss}}\xspace}
\newcommand{\ptvecmiss}{\ensuremath{{\myvec{p}}_{\mathrm{T}}^{\kern1pt\text{miss}}}\xspace}
\newcommand{\ptvecvis}[1][]{\ensuremath{{\myvec{p}}_{\mathrm{T}\ifthenelse{\isempty{#1}}{}{,#1}}^{\kern1pt\text{vis}}}\xspace}
\newcommand{\MET}    {{\ensuremath{\text{MET}}}\xspace}
\newcommand{\ETmiss} {{\ensuremath{E_{\mathrm{T}}^{\text{miss}}}}\xspace}
\newcommand{\PT}     {\pt}
\newcommand{\mT}     {{\ensuremath{m_\text{T}}}\xspace}
