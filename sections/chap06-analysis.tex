%!TEX root = ../thesis.tex
\newchap{Measurement of the $|V_{cb}|$ CKM element from \ttbar semileptonic decays}\label{sec:AN}
\minitoc

This work aims to access the $|V_{cb}|$ element through the direct decay of a \PW boson at $\sqrt{s}=13 TeV$ with the recorded luminosity of the RunII.\\
An inclusive measurement of $\PW \to cb$ would be an impossible task due to the large background but is possible to exploit the semileptonic decay of top pairs and use the lepton to trigger the event and the hadronic \PW decay to measure the $\PW \to cb$ branching fraction.\\
As lepton $\ell$, we will only consider muons and electrons, due to the lower reconstruction efficiency of $\tau$s, that decay hadronically in the $64.8\%$ of cases.
\\
\\
The value of $|V_{cb}|$ can be extracted from the ratio of branching fractions in which the hadronic \PW decays into charm and beauty quarks over the total \PW hadronic branching fraction
\begin{equation}
\begin{gathered}
    R_{cb}=\frac{\Gamma\left(\ttbar \to b\ell \nu\; b cb\right)}{\Gamma\left(\ttbar \to b\ell \nu\; b q\bar{q}\right)}=\\ =\frac{|V_{cb}|^2}{|V_{ud}|^2+|V_{us}|^2+|V_{ub}|^2+|V_{cd}|^2+|V_{cs}|^2+|V_{cb}|^2}
\end{gathered}
\end{equation}
and if we assume the unitarity of the CKM matrix
\begin{equation}
    R_{cb}=\frac{|V_{cb}|^2}{2}
\end{equation}

The determination of $|V_{cb}|$ through direct \PW decays gives us the opportunity to provide a new determination of $|V_{cb}|$ at a new energy scale ($\sqrt{s}=m_{\PW}$), avoiding complications and theoretical uncertainties due to non-perturbative QCD.\\
The statistic provided by the RunII does not allow to reach the current precision of $\sim2 \%$ on $|V_{cb}|$ that was obtained observing the semileptonic decays of $B$ mesons, but, despite that, it gives us an opportunity to tackle the $|V_{cb}|$ puzzle with a novel strategy.\\
The main sources of systematic uncertainties are the b/c tagging uncertainties, but the ratio of branching fractions should allow some cancellations.\\
\\
The measurements of $|V_{cb}|$ through semileptonic B meson decays are discussed in sec. \ref{sec:vcb}.

\section{Event signatures}
\TODO{Change name of the section}\\
In this chapter, we will name "signal" or "cb events" the $\ttbar \to b\ell\nu\: bcb$ process, and "background" everything else.\\
\\
The final state of a signal event consists of one muon or electron, three b jets, one c jet, and some missing energy due to the presence of the neutrino.\\
The invariant mass of one b jet and the c jet should match the mass of the \PW boson, and, adding another b jet, should match the mass of the top quark.\\
This final state can be mimicked by the following processes.
\begin{itemize}
    \item Semileptonic $\ttbar$ ($\ttbar\to b\ell\nu \: bq\bar{q}$): if one or more jets are mistagged and/or there are additional jets.\\
    This background depends mainly on the b/c tagging capabilities of the experiment.
    \item Dileptonic $\ttbar$ ($\ttbar \to b\bar{\ell}\nu \: \bar{b}\ell\bar{\nu}$): the presence of two leptons in the final states increase the trigger probability, while the two missing jets can be originated from additional jets produced by initial state radiation (ISR) of final state radiation (FSR), or from the leptons.
    \item Dihadronic $\ttbar$ ($\ttbar \to b\bar{q}q \: \bar{b}q\bar{q}$): the missing lepton can be originated by the semileptonic decay of a meson in the jet if it passes the isolation selections, and the missing b jet can be obtained by mistagging or by additional heavy flavor jets.
    \item W+jets: if the \PW decay into a lepton. The additional jets must be three b jets and one c jets but the production of additional heavy flavor jets is highly suppressed 
    \item WW+jets: the semileptonic decay of \PW pairs produced directly by $pp$ collision, in association with two other b jets, has the same signature of the signal. Another difference with the signal is the mismatch of the invariant mass of 3 jets with the top mass, which, however, is difficult to exploit due to the combinatorial ambiguities and the low tri-jet mass resolution.  
    \item $t\PW+\text{jets}$: it is similar to the $\PW \PW + \text{jets}$ process, with the difference that only one additional b jet instead of two is required to match the signal signature.
    \item $tq$ (t-channel): if the top quark decays into a lepton. There are two missing b jets and, if the quark is not a charm quark, also a c jet.
\end{itemize}
The production cross-sections of these processes  at $\sqrt{s}=13\TeV$ with $pp$ collisions are reported in Tab.\ref{tab:cross}

\begin{table}[H]
    \centering
    \fontsize{9.2pt}{9.2pt}\selectfont
    \begin{tabular}{l|cccc|c|c|c|c}
        \toprule
          \multicolumn{1}{c|}{$\mathbf{pp\to}$}&\multicolumn{4}{c|}{$ \mathbf{t\bar{t}}$}&  $ \mathbf{W}$& $ \mathbf{WW}$ & $ \mathbf{tW}$& $ \mathbf{tq}$\\
          &&  &  &  &  &   & & (t-channel)\\
          \midrule
          \multicolumn{1}{c|}{$\mathbf{\sigma (pb)}$}&\multicolumn{4}{c|}{\multirow{2}{*}{$832$}}& \multirow{2}{*}{$59100$} & \multirow{2}{*}{$118$} &  \multirow{2}{*}{79}& \multirow{2}{*}{214} \\
          \multicolumn{1}{c|}{$(13\TeV)$}& &  & & &  &&&\\
          \midrule
          &signal&  semiLept&  diLept&  diHad& Lept &  semiLept & semiLept& Lept\\
          &$(b\ell \nu\: bcb)$&$(b\ell \nu\: b\bar{q}q)$&$(b\bar{\ell} \nu\: \bar{b}\ell \bar{\nu})$&$(\bar{b}\bar{q}q\: b\bar{q}q)$&$(\ell \nu)$&$(\ell \nu \: q\bar{q})$& $(b\ell\nu q \bar{q})$&$(b\ell\nu \: q)$\\
          \midrule
          $\mathcal{BR}$& $3.7 \cdot 10^{-4}$   & 0.439 & 0.106 & 0.455 & 0.326 & 0.106 & 0.439 & 0.326\\
          $ \mathcal{BR}\cdot\mathbf{\sigma} (pb)$& 0.307 & 365 & 88 & 379 & 19200 & 12.5 & 34.7& 69.8 \\
          $\mathcal{BR}\cdot\mathcal{L}_I\mathbf{\sigma} $&$4.2 \cdot 10^4$& $5.0 \cdot 10^7$ &  $1.2 \cdot 10^7$&$5.2 \cdot 10^7$  &  $2.6 \cdot 10^9$ & $1.7 \cdot 10^6$  & $4.8 \cdot 10^6$& $9.6 \cdot 10^{6}$\\
          \bottomrule
    \end{tabular}
    \vspace{0.2cm}
    \caption{Production cross-sections of signal and backgrounds at $\sqrt{s}=13\TeV$. In the first row, there are the production cross-sections from pp collisions at $13 \TeV$, in the following rows there are the respective branching fractions of the different final states, the cross-section of the respective final states, and the total number of events expected in all the Run2, with a total integrated luminosity of $\mathcal{L}_I=138 {fb}^{-1}$}
    \label{tab:cross}
\end{table}

\TODO{Aggiungi qualche plot LHE almeno su btagging e num jet, num lept. Oppure mettili nel capitolo della selection}
\paragraph*{Analysis strategy}
In the following sections, we will describe the selection criteria for physics objects and events, defining different signal regions for $\ell=\mu,e$.
Therefore, after reconstructing the kinematics of the events, the signal extraction is performed through a template fit on a NN score, by exploiting multivariate analysis techniques, and also studying the relevant signatures of the events.\\
This work will be conducted only on MC simulations and the observed data will be replaced by the Asimov dataset.



\section{Simulation samples}
The generators that were used to build the MC samples are \MADGRAPH 5 (at the leading order (LO)) and aMC@NLO and \POWHEG 2 (at the next-to-leading order (NLO), including virtual corrections). In all samples, the used PDF set is NNPDF3.1 NNLO and the mass of the top quark is set to 172.5\GeV. the hadronization is handled by \PYTHIA8 exploiting the Lung string model with the \texttt{TuneCP5} set of tuning parameters \ADDREF, \ie the $\alpha_S$ value used for the simulation of the multiparton interaction (MPI)\footnote{The MPI are additional soft or semi-hard parton–parton scatterings that occur within the same hadron-hadron collision.}, hard scattering, and FSR and ISR contributions is equal to 0.118 and run according to an NLO evolution. \\
For each sample, a different number of additional partons is generated to include FSR and ISR contributions and/or virtual corrections at NLO.
\\
\\
The reconstruction and the calibration are performed with the \textit{Ultra Legacy 18} (UL18) setting, which reproduces the state of the detector in 2018.\\
All the samples were used in the NanoAOD format \ADDREF, a lightweight data format that consists only of flat ROOT NTuple.
\\
A comparison of all the used samples is reported in Table \ref{tab:samples}.
\begin{table}[H]
    
    \centering
    
    \begin{tabular*}{\linewidth}{@{\extracolsep{\fill}}cccc|c}
    \toprule
    \multirow{2}{*}{\textbf{Dataset}}&\multirow{2}{*}{\textbf{Generator}} & \textbf{Additional} & \multirow{2}{*}{$\dfrac{\mathcal{L}_I^{\text{MC}}}{\mathcal{L}_I^{\text{RunII}}}$}& \multirow{2}{*}{\textbf{Label}}  \\
    &&\textbf{Partons}& &\\
    \midrule
    \ttbar signal& \MADGRAPH (LO) & \multirow{2}{*}{3} &\multirow{2}{*}{76.5}& $\ell=\mu,e,\tau$  \\
    $(b\ell\nu \: bcb)$ &+\MADSPIN & && signal(Mu,Ele,Tau) \\    
    \midrule
    \ttbar semiLept&\multirow{2}{*}{\POWHEG (NLO)} &\multirow{2}{*}{1}&\multirow{2}{*}{2.24} & $\ell=\mu,e,\tau$   \\
    $(b\ell\nu \: bqq)$ && && semiLept(Mu,Ele,Tau)\\  
    \midrule
    \ttbar diLept&\multirow{2}{*}{\POWHEG (NLO)}  &\multirow{2}{*}{1}&\multirow{2}{*}{1.71} & \multirow{2}{*}{diLept}\\
    $(b\bar{\ell}\nu \:\bar{b}\ell\bar{\nu})$&& &\\
    \midrule
    \ttbar diHad&\multirow{2}{*}{\POWHEG (NLO)} &\multirow{2}{*}{1}&\multirow{2}{*}{0.87} &\multirow{2}{*}{diHad}\\
    $(bq\bar{q}\: \bar{b}q\bar{q})$&& &\\
    \midrule
    W+jets& \MADGRAPH (LO) &\multirow{2}{*}{4}&\multirow{2}{*}{0.02} &\multirow{2}{*}{WJets}\\
    $(W\to\ell\nu)$&+\MADSPIN &&\\
    \midrule
    WW+jets&aMC@NLO (NLO) & \multirow{2}{*}{1} & \multirow{2}{*}{0.56}& \multirow{2}{*}{WWJets}\\
    $\ell \nu \: q\bar{q}$&+\MADSPIN&&\\
    \midrule
    t\PW & & & & \multirow{2}{*}{tW}\\
    $(b\ell\nu q\bar{q})$&&&&\\
    \midrule
    tq (t-channel) & & & & \multirow{2}{*}{tq (t-chan.)}\\
    $b\ell\nu q$&&&&\\

    \bottomrule
    \end{tabular*}
    \caption{MC samples used in this work along with the respective generator used, the number of additional partons set in the generator, and the ratio between the effective luminosity of the MC sample and the integrated luminosity of the RunII. In the "Label" column there are the names that are used as labels in the rest of the thesis.}
    \label{tab:samples}
\end{table}





Simulated events are normalized to
their expected contributions using event weights 
\begin{equation}
    w_{pe}=\frac{\mathcal{L}_I \cdot \sigma_p \cdot w_{\text{gen},e} }{\sum_j w_{\text{gen},j}}
\end{equation}

where $w_{\text{gen,e}}$ is the event weight produced by the generator program and $e$ is the index of the event. 

\section{Physics objects selections}
\subsection{Muons}
\subsection{Electrons}
\subsection{Jet}
\section{Event selections}
\section{Neutrino reconstruction}
\section{Feature ranking}
\subsection{Unidimensional score}
\subsection{{$N-1,N+1$}}
\section{Jet-parton assignment}
\subsection{Leptonic bJet benchmark}
\subsection{Full kinematic reconstruction}

\section{Event classification}

\section{Signal rate extraction}
\subsection{Systematic uncertainties}
\section{Ratio extraction}
\subsection{Systematic uncertainties}