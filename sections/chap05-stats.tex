%!TEX root = ../thesis.tex
\newchap{Statistical methods}\label{sec:STAT}
\minitoc
\section{Neural networks}
\subsection{Attention mechanism}

\section{Signal extraction}
The signal extraction procedure relies on a fit of a binned distribution.
The corresponding likelihood is a product of Poisson distributions, one for each bin, that contains the parameter of interest (POI) to fit.\\
Given M regions, each represented with a histogram with $N_i$ bins, the signal and background MC prediction $s_{ij}$, $b_{ij}$, and the j-th bin content for the i-th region $n_{ij}$, the likelihood is 
\begin{equation}
    \mathcal{L}(\mu)=\prod_i^M \prod_j^{N_i} \frac{(\mu s_{ij}+b_{ij})^{n_{ij}}}{n_{ij}!} e^{-(\mu s_{ij}+b_{ij})}
\end{equation}
In this case, the POI is the signal strength $\mu$, that is estimated by finding the value that maximizes the likelihood, using the MINUIT algorithm \ADDREF.
\begin{equation*}
    \hat{\mu}=argmax_\mu \mathcal{L}(\mu)
\end{equation*}

\subsection{Systematic uncertainties}
To incorporate in the model all the relevant uncertainties (\ie luminosity and rate uncertainties, uncertainties related to the detector resolution, etc.) we add to the likelihood the so-called nuisance parameters $\theta$.\\
The likelihood becomes
\begin{equation}
    \mathcal{L}(\mu,\vec{\theta})=\prod_i^M \prod_j^{N_i} \frac{(\mu s_{ij}(\vec{\theta})+b_{ij}(\vec{\theta}))^{n_{ij}}}{n_{ij}!} e^{-(\mu s_{ij}(\vec{\theta})+b_{ij}\vec{\theta})} \prod_k  \pi_{k}(\theta_k)
\end{equation}
where $\pi_{k}$ is the prior probability distribution of the nuisance $\theta_k$.
Three types of nuisances can be identified:
\begin{itemize}
    \item \text{Normalization nuisances}: Multiplicative corrections that affect the normalization of one process (\eg the cross-section uncertainties) or all processes (\eg the luminosity uncertainties).\\
    This type of uncertainty does not change the shape of the histogram but changes the number of expected events.
\end{itemize}

\subsection{Profile Likelihood}

\subsection{Asimov dataset}

