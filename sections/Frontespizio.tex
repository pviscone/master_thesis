%!TEX root = ../thesis.tex

\newgeometry{margin=1cm}
\begin{titlepage}

    \begin{figure}[!htb]
    \vspace{30mm}
        \centering
        \includegraphics[keepaspectratio=true,scale=0.5]{fig/Frontespizio/cherubinFrontespizio.eps}
    \end{figure}


\begin{center}
    \LARGE{UNIVERSITÀ DI PISA}
    \vspace{5mm}
    \\ \large{DIPARTIMENTO DI FISICA}
    \vspace{5mm}
    \\ \LARGE{Laurea Magistrale in Fisica}
\end{center}

\vspace{15mm}
\begin{center}
    {\LARGE{\textbf{Measurement of the $V_{cb}$ \CKM element}\\ \vspace{3mm} \textbf{from \PQt \PAQt semileptonic decays at \CMS} }}
    
    % Se il titolo è abbastanza corto da stare su una riga, si può usare
    
    % {\LARGE{\bf Un fantastico titolo per la mia tesi!}}
\end{center}
\vspace{30mm}

\begin{minipage}[t]{0.47\textwidth}
	{\large{Relatore:}{\normalsize\vspace{2mm}
	\\ \large{\textbf{Prof: Paolo Azzurri}} \normalsize}}
\end{minipage}
\hfill
\begin{minipage}[t]{0.47\textwidth}\raggedleft
	{\large{Candidato:}{\normalsize\vspace{2mm} \\ \large{\textbf{Piero Viscone}}}}
\end{minipage}

\vspace{35mm}
\hrulefill
\\\centering{\large{ANNO ACCADEMICO 2022/2023}}




\end{titlepage}
\restoregeometry
