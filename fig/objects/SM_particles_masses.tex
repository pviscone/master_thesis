% Author: Izaak Neutelings (December 2021)
% Inspired by
%   https://www.sciencedirect.com/science/article/pii/S0146641019300109
%   https://iopscience.iop.org/article/10.1088/1361-6471/ab4574/pdf
\documentclass[border=3pt,tikz]{standalone}
\tikzset{>=latex} % for LaTeX arrow head
\usepackage{siunitx}
\usepackage[outline]{contour} % glow around text
\usepackage{xfp} % needed for accuracy
\usepackage{pgfplots} % for the axis environment
\pgfplotsset{compat=1.13}
\contourlength{1.1pt}

% redraw axis on top / in front (over filled areas)
\makeatletter \newcommand{\pgfplotsdrawaxis}{\pgfplots@draw@axis} \makeatother
\pgfplotsset{axis line on top/.style={after end axis/.append code={\pgfplotsdrawaxis}}}

% UNSLANT GREEK LETTERS for particle symbols
% https://tex.stackexchange.com/questions/145926/upright-greek-font-fitting-to-computer-modern
% https://tex.stackexchange.com/questions/236915/adjust-custom-made-upright-greek-letters-when-used-in-subscripts
\usepackage{scalerel}
\newsavebox{\foobox}
\newcommand{\slantbox}[2][0]{\mbox{%
  \sbox{\foobox}{#2}%
  \hskip\wd\foobox
  \pdfsave
  \pdfsetmatrix{1 0 #1 1}%
  \llap{\usebox{\foobox}}%
  \pdfrestore
}}
\newcommand\unslant[2][-.25]{%
  \mkern1.2mu%
  \ThisStyle{\slantbox[#1]{$\SavedStyle#2$}}%
  \mkern-1.2mu%
}

\colorlet{boscol}{red!80!black} % boson color
\colorlet{mescol}{blue!70!cyan!90!black} % meson color
\colorlet{barcol}{blue!80!cyan!65!red!85} % baryon color
\colorlet{lepcol}{green!80!black} % lepton color
\colorlet{promptcol}{violet!40!black} % prompt color
\colorlet{dispcol}{orange!40!black} % displaced color
\colorlet{stablecol}{green!40!black} % stable color
%\tikzstyle{particle}=[ball color=#1]
\def\r{1.3pt}
\def\part[#1](#2){
  \fill[ball color=#1] (#2) circle(\r);
  \fill[#1,opacity=0.4] (#2) circle(\r);
  \draw[#1!60!black,line width=0.1] (#2) circle(\r)
}
\tikzstyle{mytick}=[black,line width=0.6]
\tikzstyle{myminortick}=[black,line width=0.4]

\begin{document}

% LOG-LOG
\begin{tikzpicture}
  \message{^^JLog-log plot}
  \def\xmin{1e-26}
  \def\xmax{1e6}
  \def\ymin{3e-4}
  \def\ymax{2e3}
  %\pgfmathsetmacro\ctmin{\xmin*3e8} % dimension too large
  %\pgfmathsetmacro\ctmax{\xmax*3e8} % dimension too large
  \def\ctmin{3e-18} % \xmin/3e8 = 1e-26*3e8
  \def\ctmax{3e14} % \xmax/3e8 = 1e6*3e8
  \begin{loglogaxis}[
      xmin=\xmin, xmax=\xmax,
      ymin=\ymin, ymax=\ymax,
      log basis x=10,
      log basis y=10,
      axis line style={mytick},
      tick style={mytick},
      minor tick style={myminortick},
      ticklabel style={scale=0.75},
      yticklabel={% % format 10^0 as 1
        \pgfmathsetmacro\t{int(\tick)} \ifnum \t = 0 $1$ \else $10^{\t}$ \fi%
      },
      %ytick={0.0001,0.001,0.01,0.1,1,10,100,1000},
      max space between ticks=25,
      yminorticks=true,
      %xminorticks=true,
      xlabel={Proper lifetime $\tau$\:[s]},
      ylabel={Particle mass $m$\:[$\text{GeV}/c^2$]},
      xlabel style={below=-3pt},
      ylabel style={above=-7pt},
      width=8.5cm, height=6.5cm,
      axis x line*=bottom,
      axis y line*=left,
      axis line on top % redraw axis on top/in front
    ]
    
    % BANDS
    \def\xprompt{3.3e-15} % prompt decay in detector (1e-6/3e8)
    \def\xdisp{  3.3e-12} % displaced vertex (1e-3/3e8)
    \def\xstable{3.3e-8}  % stable in detector (10/3e-8)
    \fill[dispcol!3] (\xstable,\ymin) rectangle (\xprompt,\ymax);
    \fill[promptcol!7] (\xmin,\ymin) rectangle (3*\xprompt,\ymax);
    \fill[stablecol!7] (0.33*\xstable,\ymin) rectangle (\xmax,\ymax);
    \draw[densely dashed,promptcol!60] (\xprompt,\ymin) -- (\xprompt,\ymax)
      node[pos=0.05,rotate=90,above right=-0.5pt,scale=0.6] {$c\tau = \SI{1}{\micro m}$};
    \draw[densely dashed,dispcol!60] (\xdisp,\ymin) -- (\xdisp,\ymax)
      node[pos=0.05,rotate=90,above right=-0.5pt,scale=0.6] {$c\tau = \SI{1}{mm}$};
    \draw[densely dashed,stablecol!60] (\xstable,\ymin) -- (\xstable,\ymax)
      node[pos=0.05,rotate=90,below right=-0.5pt,scale=0.6] {$c\tau = \SI{10}{m}$};
    \node[below=3,scale=0.7,promptcol] at ({\xmin*10^(log10(\xprompt/\xmin)/2)},\ymax)
      {Detector-prompt};
    \node[below=3,scale=0.7,promptcol,align=center] at ({\xprompt*10^(log10(\xstable/\xprompt)/2)},\ymax) %,fill=dispcol!3
      {\contour{dispcol!3}{Displaced}\\\contour{dispcol!3}{vertex}};
    \node[below=3,scale=0.7,stablecol] at ({\xstable*10^(log10(\xmax/\xstable)/2)},\ymax)
      {Detector-stable};
    
    \begin{scope}[every node/.style={scale=0.84}]
      
      % PARTICLES: QUARKS
      \part[mescol!50!cyan](5e-25,173) node[above=0pt] {t};
      
      % PARTICLES: BARYONS
      % https://en.wikipedia.org/wiki/List_of_baryons#JP_=_1/2+_baryons
      % https://en.wikipedia.org/wiki/Lambda_baryon#Types_of_lambda_baryons
      \part[barcol](879,940e-3) node[anchor=-40,inner sep=1.6pt] {n$^0$};
      \draw[<-,thick,barcol] (\xmax,938e-3) --++ (-10pt,0pt)
        node[pos=0.85,barcol!40!black,anchor=-90,inner sep=1pt] {p$^\pm$};
      %\part[barcol](8.0e-11,1.190) node[anchor=-105,inner sep=3.5pt] {$\Sigma^+$};
      \part[barcol](7.4e-20,1.190) node[anchor=-10,inner sep=2pt] {$\Sigma^0$};
      \part[barcol](2.6e-10,1.120) node[anchor=-100,inner sep=3.5pt] {$\Lambda$};
      \part[barcol](2.0e-13,2.290) node[anchor=-60,inner sep=3pt] {$\Lambda_\mathrm{c}$};
      \part[barcol](1.5e-12,5.620) node[anchor=-60,inner sep=3pt] {$\Lambda_\mathrm{b}$};
      
      % PARTICLES: MESONS
      % https://en.wikipedia.org/wiki/Meson#List
      % https://en.wikipedia.org/wiki/B_meson#List_of_B_mesons
      \part[mescol](8.4e-17,135e-3) node[anchor=-110,inner sep=3pt] {$\pi^0$};
      \part[mescol](4.5e-24,775e-3) node[anchor=70,inner sep=4pt] {$\unslant\rho$}; %^{\pm,0}
      \part[mescol](7.8e-23,783e-3) node[anchor=110,inner sep=4pt] {$\unslant\omega$};
      \part[mescol](5.0e-19,548e-3) node[anchor=70,inner sep=3pt] {$\unslant\eta$};
      \part[mescol](2.6e-8,140e-3) node[anchor=-160,inner sep=3pt] {$\unslant\pi^\pm$};
      \part[mescol](1.2e-8,494e-3) node[anchor=-100,inner sep=3.5pt] {K\contour{stablecol!7}{$^\pm$}};
      %\part[mescol](1.2e-8,494e-3) node[anchor=35,inner sep=1pt] {K$^\pm$};
      %\part[mescol](9.0e-11,498e-3) node[anchor=35,inner sep=2pt] {\contour{dispcol!3}{K$_\mathrm{S}^0$}};
      \part[mescol](9.0e-11,498e-3) node[anchor=95,inner sep=3pt] {\contour{dispcol!3}{K$_\mathrm{S}^0$}};
      \part[mescol](5.1e-8,498e-3) node[anchor=190,inner sep=3.5pt] {K$_\mathrm{L}^0$};
      \part[mescol](7.4e-20,892e-3) node[anchor=-165,inner sep=3pt] {K$^{*\pm}$};
      \part[mescol](7.1e-21,3.10) node[anchor=190,inner sep=4pt] {J/$\unslant\psi$};
      \part[mescol](7.1e-21,9.46) node[anchor=200,inner sep=3pt] {$\Upsilon$};
      \part[mescol](1e-12,1.86)
        node[anchor=-175,inner sep=3pt] {\contour{dispcol!3}{D}}; %$^{\pm,0}$,\,D$^0$
      \part[mescol](1.6e-12,5.28)
        node[anchor=-170,inner sep=3pt] {B}; %$^{\pm,0}$,\,B$^0$
      
      % PARTICLES: LEPTONS
      % https://en.wikipedia.org/wiki/Particle_decay#Table_of_some_elementary_and_composite_particle_lifetimes
      % https://www.researchgate.net/figure/Conversion-of-21-particle-lifetimes-and-masses-to-US9_tbl1_282305954
      \part[lepcol](2.2e-6,105e-3) node[anchor=120,inner sep=3pt] {$\unslant\mu$};
      \part[lepcol](2.9e-13,1.777) node[anchor=60,inner sep=3pt] {$\unslant\tau$};
      \draw[<-,thick,lepcol] (\xmax,510e-6) --++ (-10pt,0pt)
        node[pos=0.8,lepcol!40!black,anchor=-50,inner sep=2pt] {e};
      
      % PARTICLES: BOSONS
      \part[boscol](1e-25,91,0); %node[above right=-1pt] {W$^\pm$};
      \part[boscol](1e-25,80.4) node[anchor=150,inner sep=3pt] {W$^\pm$,\,Z$^0$};
      \part[boscol](1.6e-22,125) node[above right=-1pt] {H$^0$};
      
    \end{scope}
    
    % MINOR X TICKS for every decade
    \pgfplotsinvokeforeach{-26,...,6}{
      \draw[myminortick] ({10^(#1)},{\ymin}) --++ (0pt,3.0pt);
    }
    
  \end{loglogaxis}
  
  % TOP AXIS (pc/mc^2 vs. ctau)
  % TODO: y-axis transformation m -> 1/mc^2
  \begin{loglogaxis}[
      xmin=\ctmin, xmax=\ctmax,
      %ymin=\ymin, ymax=\ymax,
      ymin=\ymax, ymax=\ymin, % reversed for 1/x!
      log basis x=10,
      log basis y=10,
      axis line style={mytick},
      tick style={mytick},
      minor tick style={myminortick},
      ticklabel style={scale=0.75},
      max space between ticks=32,
      xtick={1e-16,1e-12,1e-8,1e-4,1,1e4,1e8,1e12},
      %xtick={1e-15,1e-12,1e-9,1e-6,1e-3,1,1e3,1e6,1e9,1e12},
      xticklabel={% % format 10^0 as 1
        \pgfmathsetmacro\t{int(\tick)} \ifnum \t = 0 $\hspace{-3pt}1$ \else $10^{\t}$ \fi%
      },
      yticklabel={% % format 10^0 as 1
        \pgfmathsetmacro\t{int(\tick)} \hspace{-6pt} \ifnum \t = 0 $1$ \else $10^{\t}$ \fi%
      },
      xlabel={$c\tau$\:[m]},
      ylabel={$\gamma\beta/pc = 1/mc^2$\:[$1/\text{GeV}$]},
      xlabel style={above=-4pt},
      ylabel style={above=-14pt},
      ymajorticks=true,
      yminorticks=true,
      %scaled ticks=false,
      y coord trafo/.code={\pgfmathparse{\fpeval{1/(#1)}}\pgfmathresult}, % 1/x for tick position
      y coord inv trafo/.code={\pgfmathparse{#1}\pgfmathresult}, % 1/x for tick labels
      y dir=reverse,
      axis x line*=top,
      axis y line*=right,
      width=8.5cm, height=6.5cm
    ]
    
    % MINOR X TICKS every decade
    \pgfplotsinvokeforeach{-18,...,16}{
      \draw[myminortick] ({10^(#1)},{\ymax}) --++ (0pt,-3.0pt);
    }
    
  \end{loglogaxis}
  
  
\end{tikzpicture}


\end{document}