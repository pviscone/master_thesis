% !TEX program = pdflatexmk
% !TEX parameter = -shell-escape
% Author: Izaak Neutelings (July 2017)
\documentclass[a4paper,12pt]{article}
\usepackage[margin=2.4cm]{geometry} % margins
\usepackage{amsmath}
\usepackage{graphicx}
\usepackage{feynmp-auto}
 
\begin{document}
 
\section{Typesetting equations with Feynman diagrams}
 
This is an equation of the dominant correction to the Higgs mass squared, containing a Feynman diagram:
\begin{equation}\label{eq:hierarchy_problem}
\begin{fmffile}{higgs_mass_correction}
    \Delta m_H^2
    \,\, = \,\,\parbox{80pt}{
    \begin{fmfgraph*}(80,60)
       \fmfleft{i}
       \fmfright{o}
       \fmfv{label=H,l.a=60}{i}
       \fmfv{label=H,l.a=120}{o}
       \fmf{dashes,tension=1}{i,v1} % ,label=H,label.side=left
       \fmf{dashes,tension=1}{v2,o}
       \fmf{fermion,left,tension=0.4,label=$\text{t}$}{v1,v2,v1}
    \end{fmfgraph*}}
    \,\, + \,\,\ldots
    \,\, = \,\, -\frac{|\lambda_\text{t}|^2}{8\pi}\Lambda^2 + \ldots
\end{fmffile}
\end{equation}
 
% SUSY
To avoid finetuning in Eq.~\eqref{eq:hierarchy_problem}, one can introduce supersymmetry (SUSY), which assumes each fermion has a scalar superpartner. SUSY will introduce an extra loop diagram: \vspace{5mm}
\begin{equation}\label{eq:hierarchy_problem_SUSY}
  \begin{fmffile}{higgs_mass_correction_SUSY}
    \Delta m_H^2
    \,\, = \,\,\parbox{80pt}{
    \begin{fmfgraph*}(80,60)
       \fmfleft{i}
       \fmfright{o}
       \fmfv{label=H,l.a=60}{i}
       \fmfv{label=H,l.a=120}{o}
       \fmf{dashes,tension=1}{i,v1} % ,label=H,label.side=left
       \fmf{dashes,tension=1}{v2,o}
       \fmf{fermion,left,tension=0.4,label=$\text{t}$}{v1,v2,v1}
    \end{fmfgraph*}}
    \,\, + \,\,\parbox{80pt}{
    \begin{fmfgraph*}(80,60)
       \fmfleft{i}
       \fmfright{o}
       \fmftop{m}
       \fmfv{label=H,l.a=60}{i}
       \fmfv{label=H,l.a=120}{o}
       \fmflabel{$\widetilde{\text{t}}$}{m}
       \fmf{dashes,tension=1}{i,v1}
       \fmf{dashes,tension=1}{v1,o}
       \fmf{dashes,right,tension=0}{v1,m,v1}
    \end{fmfgraph*}}
    \,\, + \,\,\ldots
  \end{fmffile}
\end{equation}
 
% VLQ
Yet another solution to finetuning are models with vector-like quarks (VLQs): \vspace{5mm}
\begin{equation}\label{eq:hierarchy_problem_VLQ}
\begin{fmffile}{higgs_mass_correction_VLQ}
    \Delta m_H^2
    \,\, = \,\,\parbox{80pt}{
    \begin{fmfgraph*}(80,60)
       \fmfleft{i}
       \fmfright{o}
       \fmfv{label=H,l.a=60}{i}
       \fmfv{label=H,l.a=120}{o}
       \fmf{dashes,tension=1}{i,v1} % ,label=H,label.side=left
       \fmf{dashes,tension=1}{v2,o}
       \fmf{fermion,left,tension=0.4,label=$\text{t}$}{v1,v2,v1}
    \end{fmfgraph*}}
    \,\, + \,\,\parbox{80pt}{
    \begin{fmfgraph*}(80,60)
       \fmfleft{i}
       \fmfright{o}
       \fmfv{label=H,l.a=60}{i}
       \fmfv{label=H,l.a=120}{o}
       \fmf{dashes,tension=1}{i,v1} % ,label=H,label.side=left
       \fmf{dashes,tension=1}{v2,o}
       \fmf{fermion,left,tension=0.4,label=$\text{t}$}{v2,v1}
       \fmf{fermion,left,tension=0.4,label=$\text{T}$}{v1,v2}
    \end{fmfgraph*}}
    \,\, + \,\,\parbox{80pt}{
    \begin{fmfgraph*}(80,60)
       \fmfleft{i}
       \fmfright{o}
       \fmftop{m}
       \fmfv{label=H,l.a=60}{i}
       \fmfv{label=H,l.a=120}{o}
       \fmflabel{$\text{T}$}{m}
       \fmf{dashes,tension=1}{i,v1}
       \fmf{dashes,tension=1}{v1,o}
       \fmf{fermion,right,tension=0}{v1,m,v1}
    \end{fmfgraph*}}
    \,\, + \,\,\ldots
\end{fmffile}
\end{equation}
 
Notice that in the code, \verb|\,| was used to add extra space between the diagrams and math symbols. Alternatively, you can use the wider \verb|\quad|.
 
Because the stop label $\widetilde{\text{t}}$ and VLQ top partner T labels were sticking out, an extra \verb|\vspace{5mm}| was needed above the equation to prevent these label from overlapping with the line above.\\
 
Equations with \verb|feymp| diagrams might not compile in the \verb|align| environment. Instead, use \verb|aligned| within the \verb|equation| environment. For example:
\begin{equation}
\begin{aligned}
  \Delta m_H^2 \,\, &=
  \begin{fmffile}{higgs_mass_correction}
    \,\,\parbox{80pt}{
    \begin{fmfgraph*}(80,60)
       \fmfleft{i}
       \fmfright{o}
       \fmfv{label=H,l.a=60}{i}
       \fmfv{label=H,l.a=120}{o}
       \fmf{dashes,tension=1}{i,v1} % ,label=H,label.side=left
       \fmf{dashes,tension=1}{v2,o}
       \fmf{fermion,left,tension=0.4,label=$\text{t}$}{v1,v2,v1}
    \end{fmfgraph*}}
    \,\, + \,\,\ldots
  \end{fmffile}\\ %[10pt] % to add extra vertical white space between the line
   &= \,\, -\frac{|\lambda_\text{t}|^2}{8\pi}\Lambda^2 \,\, + \,\,\ldots
\end{aligned}
\end{equation}
 
Another tip: use \verb|\\[10pt]| instead of just \verb|\\| at the end of a line in the \verb|aligned| environment to add extra vertical white space between two lines of equations.
 
\end{document}